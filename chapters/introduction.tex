\chapter{Introduction}

Recruitment and retention of military personnel have become an increasingly important issue for many NATO countries due to demographic changes, socioeconomic conditions and geopolitical tensions \parencite{nato_research_and_technology_organization_recruiting_2007,nato_nato_2022}.
Threats like Russia's invasion of Ukraine emphasise the need for sufficient manpower in NATO members' armed forces. 
At the same time, according to \textcite{nato_research_and_technology_organization_recruiting_2007}, reports of large proportions of recruits leaving the military during their first term have not been unusual.
Understanding the factors that influence military labour supply is crucial for providing advice to NATO policy makers to improve defence planning.

This thesis addresses the following research question: \textit{How do unemployment rate, GDP, defense spending, and educational attainment affect the number of active military personnel in NATO countries?}
It aims to provide empirical results on how broader economic conditions may affect military labour supply and thus support defense policy formulation.

Previous studies have shown that higher unemployment rates are often linked to increased enlistment, for example, in Sweden and the United States, while findings from the Czech Republic have suggested more complex relationships \parencite{backstrom_are_2019,asch_cash_2010,holcner_military_2021}. 
GDP and defense spending have been studied as signals of a country's civilian and military economy, which influences the choice of military employment, though results remain mixed \parencite{warner_chapter_1995,holcner_military_2021}. 
Furthermore, educational attainment is seen as both a requirement for modern armed forces seeking highly skilled personnel and a challenge for personnel retention, as better-educated individuals may perceive better opportunities in the civilian sector \parencite{cnas_resources_and_force_readiness_division_fiscal_nodate,hof_quality_2023}.

Crucially, existing research often focuses only on individual countries. 
These findings are context-specific and not easily applicable across NATO as a whole. 
Additionally, previous studies tend to analyse factors in isolation rather than as part of a multi-variable system. 
This approach does not account for the potential combined effects of the variables. 
As a result, the current knowledge lacks a comprehensive understanding of how multiple variables jointly influence military size. 
Current literature also provides conflicting findings, leading to uncertainty about the exact effects of various socioeconomic factors. 
This thesis aims to fill those gaps in the literature by taking a broader approach and empirically analysing the effect of multiple variables across all NATO countries and over several years.
It hopes to find generalizable results, which NATO policymakers could use across all NATO countries for enhancing the defence planning strategy.

Following this introduction, Chapter 2 reviews the existing literature for links between socioeconomic conditions and military labour supply, and methods used to study those links. 
Chapter 3 describes the context of this study, connecting it with NATO's strategy and challenges. Chapter 4 outlines the methodology used for collecting the data and analysing it, while Chapter 5 presents the results of the analysis.
Finally, Chapter 6 concludes the findings and presents future research opportunities.