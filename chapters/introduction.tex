\chapter{Introduction}

Recruitment and retention of military personnel have become an increasingly 
important issue for many NATO countries due to demographic changes, socioeconomic 
conditions and geopolitical tensions \parencite{nato_research_and_technology_organization_recruiting_2007,nato_nato_2022}.
Understanding the factors that influence military labour supply is crucial for providing 
advice to NATO policy makers to improve defence planning \parencite{nato_research_and_technology_organization_recruiting_2007}.
This thesis investigates the relationships of
unemployment rate, GDP, defense spending, and educational attainment with
the number of active military personnel in NATO countries.
It aims to provide empirical results on how broader economic conditions 
may affect military labour supply and thus support defense policy formulation.

Previous studies have shown that higher unemployment rates are often 
linked to increased enlistment, for example, in Sweden and the United States, 
while findings from the Czech Republic have suggested more complex relationships
\parencite{backstrom_are_2019,asch_cash_2010,holcner_military_2021}. 
GDP and defense spending have been studied as signals of a country’s 
civilian and military economy, which influences the choice of military employment, 
though results remain mixed \parencite{warner_chapter_1995,holcner_military_2021}. 
Furthermore, educational attainment is seen as both a requirement for modern armed forces seeking highly skilled personnel 
and a challenge for personnel retention
\parencite{cnas_resources_and_force_readiness_division_fiscal_nodate,hof_quality_2023}. 
While existing research 
often focuses on individual countries or isolated variables; this thesis takes a 
broader approach by analysing data across multiple NATO countries and over several years, 
aiming to fill a notable gap in the literature and find generalizable results.