\chapter{Results}

\section{Correlation Analysis}

The correlation analysis revealed several significant and unsignificant relationships 
between the socioeconomic indicators and the number of active military personnel.

First, a strong positive correlation was found between the log transformed 
active military personnel and the log transformed population (0.96, p < 0.001).
Population was added as a control variable as it was expected that countris with 
larger populations also have potential for a a larger military. 
A strong positive correlation was 
also found for the log transformed GDP (0.85, p < 0.001), which also makes sense 
as wealthier countries should be able to afford larger militaries.

A significant negative correlation was discovered for the annual change in defence 
spending per capita (-0.36, p = 0.037). This suggested that countries where defence spending 
per capita increased from the previous year tended to have a smaller active military force.
A negative correlation, although not significant, was also found with unemployment rate 
(-0.25, p = 0.17) and the annual change in defence spending as a percentage of GDP 
(-0.29, p = 0.11).


\section{Regression Analysis}