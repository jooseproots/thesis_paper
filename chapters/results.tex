\chapter{Results}

\section{Correlation Analysis}

The correlation analysis revealed several significant and unsignificant relationships 
between the socioeconomic indicators and the number of active military personnel.

First, a strong positive correlation was found between the log transformed 
active military personnel and the log transformed population $(0.96, p < 0.001)$.
Population was added as a control variable as it was expected that countris with 
larger populations also have potential for a a larger military. 
A strong positive correlation was 
also found for the log transformed GDP $(0.85, p < 0.001)$, which also makes sense 
as wealthier countries should be able to afford larger militaries.

A significant negative correlation was discovered for the annual change in defence 
spending per capita $(-0.36, p = 0.037)$. This suggested that countries where defence spending 
per capita increased from the previous year tended to have a smaller active military force.
A negative correlation, although not significant, was also found with unemployment rate 
$(-0.25, p = 0.17)$ and the annual change in defence spending as a percentage of GDP 
$(-0.29, p = 0.11)$.


\section{Regression Analysis}

\subsection{Sensitivity Analysis}

A sensitivity analsis was conducted to determine whether to include interpolated and filled educational 
attainment data in the regression model. Three regression models were created: Model A with 
complete data, including observations with interpolated educational attainment values, Model B 
that excluded rows with interpolated values and Model C with complete data and an additional 
dummy variable indicating whether the education data was interpolated or not.

\renewcommand{\arraystretch}{1.3}

\begin{table}[ht]
\small
\caption{Sensitivity analysis models}
\centering
\resizebox{\textwidth}{!}{%
\begin{tabularx}{\textwidth}{>{\hsize=0.85\hsize\raggedright\arraybackslash}X 
                            >{\hsize=1\hsize\raggedright\arraybackslash}X 
                            >{\hsize=1\hsize\raggedright\arraybackslash}X 
                            >{\hsize=1\hsize\raggedright\arraybackslash}X}
\toprule
\textbf{Term} & \textbf{Model A (full)} & \textbf{Model B (excl. filled)} & \textbf{Model C (full + dummy)} \\
\midrule
\multicolumn{4}{l}{\textit{Coefficient estimates}} \\
Secondary education attainment rate & $-0.0075, p=0.0001$ & $-0.0054, p=0.0078$ & $-0.0076, p=0.0001$ \\
Interpolation dummy & - & - & $-0.0536, p=0.0371$ \\
\addlinespace
\multicolumn{4}{l}{\textit{Model statistics}} \\
Number of observations & 285 & 262 & 285 \\
$R^2$ (within) & 0.2066 & 0.2194 & 0.2206 \\
F-statistic (robust) & $F(8,245)=7.97, p<0.001$ & $F(8,222)=7.80, p<0.001$ & $F(9,244)=7.68, p<0.001$ \\
\bottomrule
\end{tabularx}
}
\label{tab:sensitivity}
\end{table}

The secondary educational attainment rate coefficents stayed consistently negative and statistically
significant across models. Additionally, other coefficients also remained with similar magnitudes 
and significances. This means that using interpolated values does not drastically distort the 
relationship with the target variable. However the dummy variable for interpolated values in Model C 
is significant, suggesting systematic differences in observations, where the education 
rate was interpolated or filled in. The negative coefficient indicates that interpolated observations 
tends to have slightly lower values of dependent variable, holding everything else constant.

Based on these results, Model B was chosen as the prefferred regression model. Imputed data could 
create noise or subtle bias, which was supported by the fact that the interpolation dummy was 
found to be statistically significant. Additionally, Model B had a marginally larger within country 
$R^2$, meaning it explained the variation within countries slightly better than Model A. A limitation 
of choosing Model B is that it has less observations, which may weaken the statistical power, 
however the loss of statistical power seemed modest.

% /subsection{Fixed-Effects regression}