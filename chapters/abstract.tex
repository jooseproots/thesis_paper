\chapter{Abstract}

This thesis examines how key economic indicators correlate with the number of active 
military personnel across NATO countries. As many NATO member states are faced 
with difficulties in military recruitment and retention, understanding the socioeconomic 
drivers behind these issues is critical for defence planning. 
The study focuses on four indicators: unemployment rate, GDP, defense spending, 
and educational attainment to assess their relationships with military labour supply.

A panel dataset covering multiple years across NATO countries was compiled using 
data from The Military Balance and the World Bank. 
The analysis includes a correlation study and a 
fixed-effects regression model to control for country-specific heterogeneity.

The results show ...

The thesis contributes to a better understanding of military labour dynamics 
and highlights the need for further cross-country, 
multi-variable research in this field.