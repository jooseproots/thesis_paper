\chapter*{Abstract}

This thesis examines how key socioeconomic indicators \textemdash\ unemployment rate, GDP, defence spending 
and educational attainment \textemdash\ influence the number of active military personnel in NATO countries.
It is motivated by recruitment and retention challenges faced by NATO countries
amid increasing geopolitical tensions and shifting defence priorities.
The study compiles a panel data set covering years 2015-2023 across NATO countries 
and analyses it using a correlation analysis and a 
fixed-effects regression model. The research controls for country-specific heterogeneity, to
assess the individual effects of these socioeconomic factors on active military labour supply.
The results indicate that a higher share of defence spending in a country's GDP is associated 
with increased military personnel, while a rising secondary education rate is related to decreases 
in military labour supply. 
This study contributes to a better understanding of military labour dynamics, 
provides evidence-based insights for NATO policymakers seeking to improve defence planning,
and highlights the need for further cross-country, 
multivariate research in this field.