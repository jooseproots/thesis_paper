\chapter{Conclusion}

This thesis set out to investigate how socioeconomic factors like unemployment rate, educational 
attainment, GDP and defence spending influence active military personnel size in NATO member countries 
from 2015 to 2023. The analysis was concluded on a panel dataset, using a fixed-effects regression model.
Several findings were discovered for academic research and defence policy.

One of the most important and statistically significant finding was that a higher share of defence 
spending in a country's GDP is associated with a larger active armed forces size per capita. Specifically, 
a 1 percentage point increase in defence spending's proportion in GDP was found to correspond with 
approximately an 8.8\% increase in active military size per capita.
This is particularly valuable as the share of defence spending in GDP is currently a relevant and 
widely discussed metric in the context of rising geopolitical tensions and shifting defence priorities.
In 2014 NATO Heads of State and Government commited to meeting a threshold of 2\% defence spending's 
proprtion in GDP and in 2025 all NATO countries are expected to meet or exceed this target \parencite{nato_defence_2025}. 
According to the findings, this increase could also result in larger active armed forces.

Secondary education attainment rates were significantly and negatively related to active military 
size per capita. A percentage point increase in the attainment rate was associated with a decrease 
of about 0.84\% in active military size per capita. This finding could mean that military labour 
supply may decrease as populations become more educated. The results align with \textcite{hof_quality_2023}, 
who found better educated military reqruits to have higher intentions to quit basic training.
The reason for this negative relationship could be for example better opportunities in the civilian 
sector for more educated individuals or the shift towards more technology versus manpower focused 
militaries.

Contrary to some prior studies, unemployment rate did not exhibit statistically significant 
relationships with active armed forces numbers. Military service is often regarded as a stable
employment opportunity, especially in bad labour market conditions, however this study found 
no evidence of labour market dynamics influencing military size. Furthermore, GDP per capita was 
also found insignificant, also confirming that borader economic conditions alone may not be the 
key drivers of military labour supply.

The annual changes in GDP per capita and defence spending's share in GDP were also found to be 
insignificant predictors of active military size. Short-term changes in GDP per capita may be 
too volatile to have immidiate effects on active military personnel, as defence planning could 
follow more long-term frameworks. The year-over-year changes in defence spending's proportion 
of GDP being insignificant while the base proportion of defence spending in GDP is significant 
could also reflect the fact that a long-term commitment to a higher defence spending is 
necessary to increase military manpower, while short-term changes may be used for
investments in other assets and do not immidiately translate to a larger military size.

For defence policy recommendations, the current study suggests keeping priority on sustained 
investments in the defence sector, as long-term commitments to a larger proportion of defence 
spending in GDP had a positive impact on military size. It also recommends considering and 
addressing the reasons why educational attainment might negatively impact military labour supply. 
New reqruitment and retention strategies could be adapted in order to appeal to more educated 
populations.

A limitation of this study is that the estimated model only explains about 12\% of within-country 
variation in active armed forces per capita. 
The aim of the study was to investigate the relationships of the selected variables on 
active military size, which it successfully did, however the model does not completely predict 
active armed forces per capita. For future studies looking to create a better prediction model, 
more variables and the for example the effect of defence planning policy or techonogical evolution of militaries 
should be taken into account as 
personnel size may not depend only on socioeconomic factors alone.
Another limitation is the sample size. According to \textcite{harrell_multivariable_2015},
in order estimate a reliable multivariate regression, the data should have at least 
10-20 observations per estimated parameter, which this model did. 
However to increase the statistical 
power of a fixed-effects regression, more observations per group/country could be used.
Future research could collect a larger sample of data for more explanatory power or 
include additional countries for more generalizable results.
Another research opportunity might be investigating the reasons, why exactly educational 
attainment has a negative correlation to active military size, as this study does not reveal 
the causal effects.

