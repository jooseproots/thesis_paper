\chapter{Conclusion}

This thesis set out to investigate how socioeconomic factors like unemployment rate, educational 
attainment, GDP and defence spending influence active military personnel size in NATO member countries 
from 2015 to 2023. The analysis was conducted on a panel dataset, using a fixed-effects regression model.
Several findings were discovered for academic research and defence policy.

One of the most important and statistically significant findings was that a higher share of defence 
spending in a country's GDP is associated with a larger active armed forces size per capita.
This is particularly valuable as the share of defence spending in GDP is currently a relevant and 
widely discussed metric in the context of rising geopolitical tensions and shifting defence priorities.
In 2014, NATO Heads of State and Government committed to meeting a threshold of 2\% defence spending in GDP, and in 2025, all NATO countries are expected to meet or exceed this target \parencite{nato_defence_2025}. 
According to the findings, this increase could also result in larger active armed forces.

Secondary education attainment rates were significantly and negatively related to active military 
size per capita. This finding could mean that the military labour 
supply may decrease as populations become more educated. The results align with \textcite{hof_quality_2023}, 
who found that better-educated military recruits have higher intentions to quit basic training.
This negative relationship may reflect, for example, better civilian opportunities 
for more educated individuals, or a shift toward more technology-oriented, rather than 
manpower-intensive armed forces.

Contrary to some prior studies, unemployment rate did not exhibit statistically significant 
relationships with armed forces personnel. Military service is often regarded as a stable
employment opportunity, especially in bad labour market conditions; however, this study found 
no evidence of labour market dynamics influencing military size. In contrast, GDP per capita, 
which can indicate better labor market conditions, as it showed a fairly strong negative 
correlation with unemployment rates, was 
found to be statistically significant and positively related to active military size. 
Other than labour market conditions, this finding may also reflect that 
wealthier countries can afford to maintain a larger military force, because they 
have more resources available for defence spending. 

The annual changes in GDP per capita and defence spending's share in GDP were also found to be 
insignificant predictors of active military size. Short-term changes in GDP per capita may be 
too volatile to have direct effects on active military personnel, as defence planning could 
follow more long-term frameworks. The insignificance of annual changes in defence spending's 
proportion of GDP, contrasted by the significance of its overall level, may 
reflect that a sustained commitment to higher defence spending is 
necessary to grow military manpower, while short-term fluctuations may fund
investments in other assets and therefore have limited immediate impact on military size.

For defence policy recommendations, the current study suggests keeping priority on sustained 
investments in the defence sector, as long-term commitments to a larger proportion of defence 
spending in GDP had a positive impact on military size. 
Additionally, GDP per capita having a significant positive relationship 
with armed forces per capita highlights the role of broader economic development in 
supporting military capacity. This suggests that defense planning should account for 
rising national income supporting maintenance of larger militaries .
It also recommends considering and 
addressing the reasons why educational attainment might negatively impact the military labour supply. 
New recruitment and retention strategies could be adapted to appeal to more educated 
populations.

A limitation of this study is that the independent variables themselves only explained about 
20\% of the variation in active armed forces per capita with the fixed effects removed, 
while including fixed effects explained around 96\%.
This study aimed to investigate the relationships among the selected variables on 
active military size, which it successfully did; however, the variables without the fixed effects 
do not predict a large portion of the variation in  
active armed forces per capita. For future studies looking to create a better prediction model, 
more variables, and for example, the effect of defence planning policy or technological evolution of militaries 
should be taken into account as 
personnel size may not depend only on socioeconomic factors.
Another limitation is the sample size. According to \textcite{harrell_multivariable_2015},
in order to estimate a reliable multivariate regression, the data should have at least 
10-20 observations per estimated parameter, which this model did. 
However, to increase the statistical 
power of a fixed-effects regression, 
future research could collect more observations per country or 
include additional countries for more generalizable results.
An additional future research opportunity might be investigating the reasons why exactly educational 
attainment has a negative correlation to active military size, as this study does not reveal 
the causal effects.

