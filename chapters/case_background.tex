\chapter{Case Background}

NATO (North Atlantic Treaty Organization) is a security
alliance established in 1949 \parencite{noauthor_about_NATO_nodate}. As of 2025, NATO
consists of 32 member countries from North America and Europe \parencite{noauthor_member_countries_nodate}.
Its primary mission is to ensure collective freedom and security of its members through 
political and military means \parencite{noauthor_about_NATO_nodate}. Among others, NATO currenyly faces 
threats such as  Russia's invasion of Ukraine, China's growing ambitions and conflicts 
in the Middle East and Africam highlighting the need to further strengthen its deterrence 
and defence capabilities \parencite{noauthor_strategic_concept_nodate}.

It has become increasingly difficult in many NATO countries to recruit new 
and retain existing qualified military personnel. This can be attributed to 
a variety of factors, including socioeconomic conditions. \parencite{nato_research_and_technology_organization_recruiting_2007}
In light of this, this thesis explores how a selection of economic indicators 
influence the number of active military personnel in NATO countries.

The indicators unemployment rate, GDP and defence spending dynamics and educational 
attainment were selected for this thesis. The data for unemployment, GDP and educational 
attainment were obtained from the \textcite{noauthor_world_bank_nodate} and the data for defence spending 
and active military personnel were sourced from \textcite{noauthor_military_balance_nodate}. The data 
was collected for the years (timeframe!!!!!!!!) across all NATO countries (and russia and china??????). 
Data was then harmonized to create a dataset suitable for analysis.
