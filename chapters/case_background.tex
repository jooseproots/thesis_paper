\chapter{Case background}

Established in 1949, NATO (North Atlantic Treaty Organization) is a security
alliance, currently
comprising of 32 member countries from North America and Europe \parencite{nato_what_nodate, us_mission_to_nato_about_nodate}.
Its primary mission is to ensure collective freedom and security of its members through 
political and military means \parencite{us_mission_to_nato_about_nodate}. Among others, NATO currently faces 
threats such as  Russia's invasion of Ukraine, China's growing ambitions and conflicts 
in the Middle East and Africa, highlighting the need to further strengthen its deterrence 
and defence capabilities \parencite{nato_nato_2022}.

The events of September 11, 2001 led to a pivotal shift in global security. They altered the nature 
of perceived threats and military engagements. In response, most European countries redefined their 
military structures, abolishing large conscript armies in favour of all-voluntary active forces. 
This change led to the professionalization of military in many countries. \parencite{herranen_professional_2004}
With the end of conscription in numerous countries, the size of the active military force 
became a critical measure of defence capacity, which is why active armed forces size 
is the key focus of this thesis.

It has become increasingly difficult in many NATO countries to recruit new 
and retain existing qualified military personnel. 
It not uncommon for more than 30\% of military recruits to leave on their first term of service, 
especially in specialist roles that are hard and costly to recruit and train.
This can be attributed to 
a variety of factors, including socioeconomic conditions. \parencite{nato_research_and_technology_organization_recruiting_2007}
In light of this, this thesis explores how a selection of economic indicators 
influence the number of active military personnel in NATO countries.

The indicators investigated in this thesis were the unemployment rate, GDP, defence spending dynamics, 
and educational attainment. The data was collected from the \textcite{noauthor_world_bank_nodate} 
and the \textcite{noauthor_military_balance_nodate}
for years 2015-2023 across all NATO countries. Russia was included into the dataset to gain insight 
into its recruitment success during the invasion of Ukraine.

This study is relevant for NATO defence policymakers and strategic planners as it identifies how 
these socioeconomic indicators influence active military size. The findings of this thesis can 
support data-driven decision making for recruitment and retention policies. As NATO continues to 
adapt to new geopolitical circumstances, understanding the socioeconomic drivers of military 
personnel supply is valuable for maintaining operational readiness.
