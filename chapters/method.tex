\chapter{Methodology}

\section{Data Preparation}

\renewcommand{\arraystretch}{1.3}

\begin{table}[ht]
\small
\caption{Variable Descriptions}
\centering
\resizebox{\textwidth}{!}{%
\begin{tabularx}{\textwidth}{>{\hsize=1\hsize\raggedright\arraybackslash}X 
                            >{\hsize=1.75\hsize\raggedright\arraybackslash}X 
                            >{\raggedright\arraybackslash}X 
                            >{\hsize=0.6\hsize\raggedright\arraybackslash}X 
                            >{\hsize=0.6\hsize\raggedright\arraybackslash}X}
\toprule
\textbf{Variable} & \textbf{Description} & \textbf{Source} & \textbf{Unit} & \textbf{Transformation} \\
\midrule
Active armed forces per capita & Number of active military personnel per capita & Military Balance, World Bank & per capita & log() \\
Unemployment rate & National unemployment rate & World Bank & \% (0-100) & none \\
Secondary education rate & Proportion of population (25+) with secondary education & World Bank & \% (0-100) & none \\
GDP per capita & Gross Domestic Product per capita & World Bank & 2015 USD & log() \\
Defence spending per capita & A country's defence expenditure divided by population & Military Balance & 2015 USD & log() \\
Defence spending \% of GDP & Defence expenditure as a share of GDP & Military Balance & \% & none \\
GDP per capita \% change & Annual percentage change in GDP per capita & World Bank (own calculation) & \% & none \\
Defence spending per capita \% change & Annual percentage change in defence spending per capita & Military Balance (own calculation) & \% & none \\
Defence spending \% GDP \% change & Annual percentage change in defence spending as \% of GDP & Military Balance (own calculation) & \% & none \\
\bottomrule
\end{tabularx}
}
\label{tab:variables}
\end{table}

The data preparation process involved several steps to create a clean and structured 
dataset that could be used for analysis. The study combines military and socioeconomic 
data for NATO countries across multiple years. Data was collected from various sources,
transformed into a consistent format and merged into a single dataset.

First, the military personnel and defence 
spending data were aquired from different issues of \textcite{noauthor_military_balance_nodate} 
as each issue contained data for a specific year. Data for active armed forces numbers, 
defence spending per capita and defence spending's share of GDP was
selected for NATO member countries from each table.
In order to ensure that the data is comparable, the effect of inflation was removed 
from defence spending per capita data by adjusting the values to 2015 USD using the Consumer Price 
Index (CPI) of different years. The CPI data was sourced from \textcite{federal_reserve_bank_of_minneapolis_consumer_nodate}. 
The tables from different years were then merged into a single long-format table, where
each row represented observations for a specific country in a specific year as this format is 
suitable for panel data analysis.

Next, the unemployment rate, GDP per capita in constant 2015 USD, population and 
secondary educational attainment data were collected from 
\textcite{noauthor_world_bank_nodate}. Data was again filtered to include 
only NATO countries from 2015 to 2023. 
The educational attainment 
data had missing values that were handled using linear interpolation, to estimate missing values 
based on neighboring data, and backward/forward filling for data points at the start or end 
of the series, which could not be interpolated. The tables were then also transformed into a long format, 
so that they could be 
merged with the military personnel and defence spending data.

A logarithmic transformation was applied to variables active armed forces per capita, GDP per capita 
and defence spending per capita. These variables can have skewed distributions, due to disproportionate influence 
from larger countries. 
Logging can mitigate this skewness, stabilize variance and allow for interpretation of the regression 
coefficients in terms of elasticities. Other variables were not logarithmically transformed as they were 
expressed as percentages, therefore logging would make interpretation difficult and distort the scales.

Finally, the military and economic data tables were merged into a complete dataset. 
The active armed forces per capita column was created by dividing active armed forces and population data.
Additional columns for annual changes in GDP per capita, defence spending per capita  
and defence spending's share of GDP were calculated by finding their percentage changes from 
the previous year's value.

\section{Data understanding}

Table~\ref{tab:descriptive_stats} reports the descriptive statistics for the variables used.
The logged dependent variable armed forces per capita showed relatively low variation with a standard 
deviation of 0.44 and a range of values between -6.55 and -4.30 which could limit the explanatory 
power of the regression model, however the regression model might still provide insights into 
smaller changes in active military size per capita.

\begin{table}[htbp]
\caption{Descriptive statistics of variables}
\renewcommand{\arraystretch}{1.2}
\begin{tabularx}{\textwidth}{l*{9}{>{\centering\arraybackslash}X}}
\toprule
 & \rotatebox{90}{\parbox{2.5cm}{Armed Forces per cap.}} 
 & \rotatebox{90}{\parbox{2.5cm}{GDP per cap.}} 
 & \rotatebox{90}{\parbox{2.5cm}{Def. budget per cap.}} 
 & \rotatebox{90}{\parbox{2.5cm}{Unemploy-\\ment rate}} 
 & \rotatebox{90}{\parbox{2.5cm}{Secondary education rate}} 
 & \rotatebox{90}{\parbox{2.5cm}{Def. budget \% GDP}} 
 & \rotatebox{90}{\parbox{2.5cm}{GDP per cap. \% change}} 
 & \rotatebox{90}{\parbox{2.5cm}{Def. budget per cap. \% change}} 
 & \rotatebox{90}{\parbox{2.5cm}{Def. budget \% GDP \% change}} \\
\midrule
\textbf{mean} & -5.71 & 7.86 & 74.50 & 10.03 & 5.71 & 1.61 & 2.18 & 4.82 & 4.57 \\
\textbf{std} & 0.44 & 4.58 & 14.64 & 0.76 & 0.81 & 0.66 & 3.67 & 18.73 & 16.39 \\
\textbf{min} & -6.55 & 2.02 & 33.60 & 8.28 & 3.44 & 0.35 & -15.21 & -41.71 & -41.03 \\
\textbf{25\%} & -5.96 & 4.83 & 69.07 & 9.52 & 5.26 & 1.14 & 0.76 & -5.61 & -3.59 \\
\textbf{50\%} & -5.79 & 6.54 & 77.78 & 9.93 & 5.81 & 1.44 & 2.27 & 1.89 & 1.48 \\
\textbf{75\%} & -5.46 & 9.41 & 86.00 & 10.70 & 6.29 & 1.98 & 4.37 & 10.16 & 10.67 \\
\textbf{max} & -4.30 & 26.40 & 95.29 & 11.59 & 7.60 & 3.82 & 13.65 & 100.83 & 109.23 \\
\bottomrule
\end{tabularx}
\label{tab:descriptive_stats}
\end{table}

Defence spending per capita, its annual change and the annual change in defence spending's 
share of GDP however showed higher volatility, with standard deviations of 14.64, 18.73 and 
16.39 respectively. These higher standard deviations could suggest heteroskedacity. In addition 
to that, in panel data settings, standard errors are ofter correlated within entities (in this case countries), 
which means assuming independent standard errors could inflate t-statistics and potentially 
overstate statistical significance in regressions. To account for this, a fixed-effects 
regression could be estimated using standard errors clustered at the country level.

The use of imputed educational attainment values was also investigated. There were a total 
of 23 interpolated or filled values across a total of 285 observations. No country had
more than 2 imputed values in the timeframe. Given the small scale of imputed values, it was 
unlikely that a strong bias would be introduced. Nevertheless, a sensitivty analysis should 
be conducted to analyse the effect of imputation more in depth.

Outliers were detected using both Z-score (with a threshold of $\pm 3$) and IQR methods. 
Outliers were found, however comparing them to the complete dataset revealed no unrealistic 
values. Since the outliers were most likely due to high variance in some variables, not data 
quality issues, they were not excluded from the dataset.

\section{Analysis}

This section describes the analytical approach used to investigate the relationships between 
socioeconomic variables and active military personnel size.
It describes how correlations and multicollinearity of the data was evaluated, 
how a sensitivity analysis for imputed values and a robutness check for multicollinear 
variables were conducted. Finally, is is explained how the regression model was estimated.

The correlation and multicollinearity analysis were performed on demeaned data, consistent 
with the fixed-effects regression. The fixed-effects model relies on within-country variation 
over time by subtracting country-specific means, so the data was transformed similarly. For 
each observation, the mean of the variable across all year of the corresponding country was 
subtracted from the original value. This demeaning process makes the data comparable across 
countries as the variables only reflect deviations from each country's average. 

The correlation analysis involved calculating the Pearson correlation coefficient for each 
of variables, resulting in a correlation matrix. The correlation matrix was used to assess the 
direction and strength of bivariate relationships.
Variance Inflation Factors (VIF) were calculated for the independent variables to 
assess the potential presence of multicollinearity. VIF values indicate how much variance
of a coefficient estimate is inflated due to relationships with other predictors.

The regression analysis used a fixed-effects panel regression model to control for time-invariant 
heterogeneity across countries \parencite{backstrom_are_2019}. This meant that it was not 
neccessary to 
use the demeaned data, because the model already took differences between entities into account.
A fixed-effects regression isolates the within-country variation over time, enabling more accurate 
estimation of the conditional effects of economic factors on military personnel levels.
In other words, by introducing country fixed effects, the model was estimated on changes within 
countries, rather than between countries. (possibly reference backstrom!!!) 
The following equation was used to estimate the base model:
\begin{align*}
log(ArmedForces_{it}) &= \beta_1 \cdot Unemployment_{it} 
+ \beta_2 \cdot Education_{it} \\
&\quad + \beta_3 \cdot log(GDPPerCap_{it}) 
+ \beta_4 \cdot log(DefSpendingPerCap_{it}) \\
&\quad + \beta_5 \cdot DefSpending\%GDP_{it} 
+ \beta_6 \cdot \Delta GDPPerCap_{it} \\
&\quad + \beta_7 \cdot \Delta DefSpendingPerCap_{it} 
+ \beta_8 \cdot \Delta DefSpending\%GDP_{it} \\
&\quad + \alpha_i + \varepsilon_{it}
\end{align*}
where $i$ denotes the country, $t$ the time, $\Delta$ the annual change and 
$a_i$ is the country-specific fixed effect.

A sensitivity analsis was also conducted to determine whether to include interpolated and filled educational 
attainment data in the regression model. Three regression models were created: Model A with 
complete data, including observations with interpolated educational attainment values, Model B 
that excluded rows with interpolated and filled values and Model C with complete data and an additional 
dummy variable indicating whether the education data was interpolated or not.

The final fixed-effect regression model used clustered country-level standard errors to 
account for portential within-country correlation of error terms and heteroskedacity across 
countries. In panel data, observations from the same entity are often not independent, which 
violates classical OLS assumptions. Clustering allows for arbitrary correlation within 
countries, ensuring a more robust and conservative model.