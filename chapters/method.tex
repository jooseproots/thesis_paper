\chapter{Methodology}

\section{Data Preparation}

\renewcommand{\arraystretch}{1.3}

\begin{table}[ht]
\small
\caption{Variable Descriptions}
\centering
\resizebox{\textwidth}{!}{%
\begin{tabularx}{\textwidth}{>{\hsize=1\hsize\raggedright\arraybackslash}X 
                            >{\hsize=1.75\hsize\raggedright\arraybackslash}X 
                            >{\raggedright\arraybackslash}X 
                            >{\hsize=0.6\hsize\raggedright\arraybackslash}X 
                            >{\hsize=0.6\hsize\raggedright\arraybackslash}X}
\toprule
\textbf{Variable} & \textbf{Description} & \textbf{Source} & \textbf{Unit} & \textbf{Transformation} \\
\midrule
Active armed forces per capita & Number of active military personnel per capita & Military Balance, World Bank & per capita & log() \\
Unemployment rate & National unemployment rate & World Bank & \% & none \\
Secondary education rate & Proportion of population with secondary education & World Bank & \% & none \\
GDP per capita & Gross Domestic Product per capita & World Bank & 2015 USD & log() \\
Defence spending per capita & A country's defence expenditure divided by population & Military Balance & 2015 USD & log() \\
Defence spending \% of GDP & Defence expenditure as a share of GDP & Military Balance & \% & none \\
GDP per capita \% change & Annual percentage change in GDP per capita & World Bank (own calculation) & \% & none \\
Defence spending per capita \% change & Annual percentage change in defence spending per capita & Military Balance (own calculation) & \% & none \\
Defence spending \% GDP \% change & Annual percentage change in defence spending as \% of GDP & Military Balance (own calculation) & \% & none \\
\bottomrule
\end{tabularx}
}
\label{tab:variables}
\end{table}

The data preparation process involved several steps to create a clean and structured 
dataset that could be used for analysis. The study combines military and socioeconomic 
data for NATO countries across multiple years. Data was collected from various sources,
transformed into a consistent format and merged into a single dataset.

First, the military personnel and defence 
spending data were aquired from different issues of \textcite{noauthor_military_balance_nodate} 
as each issue contained data for a specific year. Columns with relevant information were 
selected from each table and renamed to ensure consistency across the dataset. A list of 
countries of interest was created to filter only NATO member countries from the tables.
Data types were then adjusted so that numerical values were presented in a consistent format.
The tables from different years were then merged into a single long-format table, where
each row represented data for a specific country in a specific year as this format is 
suitable for panel data analysis.

Next, the unemployment rate, GDP and educational attainment data were collected from 
\textcite{noauthor_world_bank_nodate}. These datasets included data across multiple years, 
so the columns with relevant years were selected and renamed for consistency. Again a 
list of countries was used to filter out only NATO countries. The educational attainment 
data had missing values that were filled using interpolation (or backwards/forwards fill) from the previous and next 
years. The tables were then also transformed into a long format, so that they could be 
merged with the military personnel and defence spending data.

Finally, in order to ensure that the data is comparable, the effect of inflation was removed 
from columns with monetary values by adjusting them to 2015 USD using the Consumer Price 
Index (CPI) of different years. The CPI data was sourced from \textcite{federal_reserve_bank_of_minneapolis_consumer_nodate}. 
The military and economic data tables were then merged into a final dataset and additional 
columns for year-on-year changes for GDP per capita, defence spending per capita  
and defence budget as a percentage of GDP were calculated using the existing data.

!!!!! Log transformations

\section{Analysis}

The analysis consisted of a correlation and multicollinearity analysis and a regression analysis. ...

The correlation and multicollinearity analysis were performed on demeaned data, consistent 
with the fixed effects regression. The fixed effects model relies on within-country variation 
over time by subtracting country-specific means, so the data was transformed similarly. For 
each observation, the mean of the variable across all year of the corresponding country was 
subtracted from the original value. This demeaning process makes the data comparable across 
countries as the variables only reflect deviations from each country's average. 

The correlation analysis involved calculating the Pearson correlation coefficient for each 
of variables, resulting in a correlation matrix. The correlation matrix was used to assess the 
direction and strength of bivariate relationships.
Variance Inflation Factors (VIF) were calculated for the independent variables to 
assess the potential presence of multicollinearity. VIF values indicate how much variance
of a coefficient estimate is inflated due to relationships with other predictors.

The regression analysis used a fixed-effects panel regression model to control for time-invariant 
heterogeneity across countries \parencite{backstrom_are_2019}. This means that it was not 
neccessary to 
use the demeaned data, because the model already takes differences between entities into account.
A fixed effects regression isolates the within-country variation over time, enabling more accurate 
estimation of the conditional effects of economic factors on military personnel levels.
In other words, by introducing country fixed effects, the model is estimated on changes within 
countries, rather than between countries. (possibly reference backstrom) 
The following model was estimated:
\begin{align*}
log(ArmedForces_{it}) &= \beta_1 \cdot Unemployment_{it} 
+ \beta_2 \cdot Education_{it} \\
&\quad + \beta_3 \cdot log(GDPPerCap_{it}) 
+ \beta_4 \cdot log(DefSpendingPerCap_{it}) \\
&\quad + \beta_5 \cdot DefSpending\%GDP_{it} 
+ \beta_6 \cdot \Delta GDPPerCap_{it} \\
&\quad + \beta_7 \cdot \Delta DefSpendingPerCap_{it} 
+ \beta_8 \cdot \Delta DefSpending\%GDP_{it} \\
&\quad + \alpha_i + \varepsilon_{it}
\end{align*}
where $i$ denotes the country, $t$ the time, $\Delta$ the annual change and 
$a_i$ is the country-specific fixed effect.

A sensitivity analsis was also conducted to determine whether to include interpolated and filled educational 
attainment data in the regression model. Three regression models were created: Model A with 
complete data, including observations with interpolated educational attainment values, Model B 
that excluded rows with interpolated and filled values and Model C with complete data and an additional 
dummy variable indicating whether the education data was interpolated or not.