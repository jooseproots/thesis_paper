\chapter{Methodology}

\section{Data Preparation}

The data preparation process involved several steps to create a clean and structured 
dataset that could be used for analysis. The study combines military and socioeconomic 
data for NATO countries across multiple years. Data was collected from various sources,
transformed into a consistent format and merged into a single dataset.

First, the military personnel and defence 
spending data were aquired from different issues of \textcite{noauthor_military_balance_nodate} 
as each issue contained data for a specific year. Columns with relevant information were 
selected from each table and renamed to ensure consistency across the dataset. A list of 
countries of interest was created to filter only NATO member countries from the tables.
Data types were then adjusted so that numerical values were presented in a consistent format.
The tables from different years were then merged into a single long-format table, where
each row represented data for a specific country in a specific year as this format is 
suitable for panel data analysis.

Next, the unemployment rate, GDP and educational attainment data were collected from 
\textcite{noauthor_world_bank_nodate}. These datasets included data across multiple years, 
so the columns with relevant years were selected and renamed for consistency. Again a 
list of countries was used to filter out only NATO countries. The educational attainment 
data had missing values that were filled using interpolation from the previous and next 
years. The tables were then also transformed into a long format, so that they could be 
merged with the military personnel and defence spending data.

Finally, in order to ensure that the data is comparable, the effect of inflation was removed 
from columns with monetary values by adjusting them to 2015 USD using the Consumer Price 
Index (CPI) of different years. The CPI data was sourced from \textcite{federal_reserve_bank_of_minneapolis_consumer_nodate}. 
The military and economic data tables were then merged into a final dataset and additional 
columns for year-on-year changes for GDP per capita, defence spending per capita  
and defence budget as a percentage of GDP were calculated using the existing data.

\section{Analysis}

The analysis consisted of a preliminary correlation analysis and a regression analysis. ...

The correlation analysis was performed on a subset of the data to identify potential 
relationships between socioeconomic indicators and the number of active military 
personnel. The subset of data included values from the year 2023 as it was the latest
year with complete data available. A log transformation was applied to variables 
population, GDP, GDP per capita, defence spending per capita and the dependent variable 
active military personnel to mitigate the effect of skewness and to more efficiently 
capture the multiplicative relationships between variables.
The Pearson correlation coefficient was calculated along with the p-values to 
determine the statistical significance of the correlation values.

The regression analysis used a fixed-effects regression model to control for heterogeneity
across countries \parencite{backstrom_are_2019}. 
