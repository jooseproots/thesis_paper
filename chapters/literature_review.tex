\chapter{Literature Review}

This literature review explores the research on the influence of socioeconomic 
factors on military labor supply. 
Recruitment and retention in the military has been studied using a wide range of
different factors. According to a report by the \textcite{nato_research_and_technology_organization_recruiting_2007}, 
the recruitment and retention issues in NATO countries are 
caused by factors such as low unemployment rates, military operational and personnel tempo,
higher civilian salaries and the shrinking pool of 18-24 year old indiciduals - among 
other challenges. External
competition for labor supply pool, recruit quality and compensation have also been 
reported as some of the reasons
individuals have expressed for not choosing a career in the military \parencite{nato_research_and_technology_organization_recruiting_2007}.
This review aims to investigate how 
factors such as GDP, unemployment, education and defence spending impact the 
attractiveness of the military as an employer. It focuses 
on quantitative studies in NATO countries in order to identify 
current knowledge of military reqruitment and retention, while also determining 
gaps in the literature that this thesis aims to address. 
First the effect of unemployment rates will be discussed, then GDP and defence spending 
and finally educational attainment.

The assertion that low unemployment rates 
cause recruitment and retention issues is supported by several authors such as 
\textcite{backstrom_are_2019} who found a positive and statistically significant correlation 
between unemployment and military application rates in Sweden and \textcite{asch_cash_2010},
who similarly found unemployment rate to be positively and significantly related to
high-quality enlistment contracts in the United States. These findings align with 
research by \textcite{balcaen_unemployment_2025} who found a one percent point increase in 
unemployment rate to result in a 0.0137 percentage point increase in military 
application rates. However, some evidence also suggests the opposite, for example 
\textcite{holcner_military_2021} found an inverse relationship between unemployment and military 
recruitment in the Czech Armed Forces.

\textcite{warner_chapter_1995} describe the decision to enlist and to remain in the military
as a choice between employment in the military or the civilian sector. The study 
mentions that a perfectly rational individual join the military sector if the pay
differential between the sectors exceeds the preference for civilian life. \textcite{warner_chapter_1995} 
also note that the USA and its allies spend a significant amount of 
their defence budgets on military personnel. This means that GDP and defence spending 
along with their dynamics could reflect employment opportunities in the civilian and 
military sector and in turn be correlated with choices to enlist and stay in the military.
Although \textcite{backstrom_are_2019} researched mostly unemployment rates, he highlights 
the fact that a stronger civilian economy increases the difficulty of recruiting new military 
labor force. On the other hand \textcite{holcner_military_2021} found that the annual 
increase in GDP, indicating a growing economy, had a positive impact on the recruitment to the 
Czech Armed Forces. The annual increase in defence expenditure was also found to be 
correlated with a higher number of military recruits \parencite{holcner_military_2021}.

The U.S Department of Defence has a benchmark of at least 90\% of new military recruits 
having secondary education or higher \parencite{cnas_resources_and_force_readiness_division_fiscal_nodate}.
This emphasizes the fact that the military is looking for educated recruits. \textcite{asoni_rich_2013} 
also argue that the transition to a smaller and technologically advanced military has 
made the recruitment process more selective and less likely for an individual to 
be allowed in the military without a high-school degree. \textcite{elster_study_1982} 
found that the the four-year retention rates were higher amongst U.S military recruits 
with a high-school degree, compared to those with lower educational attainment, however 
this study was conducted with data from the 1970s, which may not reflect the current 
socioeconomic or military environment. 
In contrast, a more recent study by \textcite{hof_quality_2023} found that in the 
Dutch Armed Forces, recruits training to become officers, who have a higher level of 
secondary education prior to enlistment, show higher intentions to quit basic training, 
than those training to become noncommissioned officers, who have a lower level of 
secondary education. The study hypothesizes that recruits with a better educational 
background could believe that they have better opportunities in the civilian labor market.

This literature review has shown that socioeconomic factors such as unemployment rates, 
GDP, defence spending and educational attainment can have a notable impact on military 
recruitment and retention. Unemployment rates are often found to be positively correlated 
with enlistment, however some evidence suggests more complex or even the opposite 
relationship. It is often assumed that a stronger civilian economy increases the 
difficulty of obtaining and maintaining military personnel and that a stonger military 
economy should attract recruits as better pay is offered, although these aspects have 
not been extensively studied. The literature also suggests that the military is 
actively recruiting individuals with a higher educational attainment, but the research 
reflects mixed results on the impact of educational attainment on personnel retention. 
The varied results and the fact that most studies focus on a single country highlights 
the need for further research across multiple NATO countries.
