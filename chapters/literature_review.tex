\chapter{Literature review}

This literature review explores the research on the influence of socioeconomic 
factors on military labour supply. 
Recruitment and retention in the military have been studied using a wide range of
different factors. According to a report by the \textcite{nato_research_and_technology_organization_recruiting_2007}, 
the recruitment and retention issues in NATO countries are 
caused by factors such as low unemployment rates,
higher civilian salaries, external
competition for the labour supply, recruit quality and the shrinking pool of 
18-24 year old individuals, among 
other challenges. This review aims to investigate how 
GDP, unemployment, education and defence spending impact the 
military's attractiveness as an employer. It focuses 
on quantitative studies in NATO countries to identify 
current knowledge on recruitment and retention, and identify 
gaps that this thesis will to address. 
First the effect of unemployment rates will be discussed, then GDP and defence spending 
and finally educational attainment.

The assertion that low unemployment rates 
cause recruitment and retention issues is supported by several authors.
For example, \textcite{backstrom_are_2019}, using a linear fixed-effects regression with panel data 
on Swedish counties in 2011-2015,
found a positive and statistically significant correlation 
between unemployment and military application rates. \textcite{asch_cash_2010} also estimated a
linear fixed-effects regression with panel data on U.S states across quarters, 
controlling for both time and state fixed effects.
They similarly found unemployment rate to be positively and significantly related to
high-quality enlistment contracts. These findings align with 
research by \textcite{balcaen_unemployment_2025} who used a linear mixed-effects regression, 
which included both fixed and random effects, to account for population and subgroup variation
in data across Belgian provinces.
This study found a one percentage point increase in 
unemployment rate to result in a 0.0137 percentage point increase in military 
application rates.
However, some evidence also suggests the opposite, for example 
\textcite{holcner_military_2021} found an inverse relationship between unemployment and military 
recruitment in the Czech Armed Forces.

\textcite{warner_chapter_1995} describe the decision to enlist and to remain in the military
as a choice between employment in the military or the civilian sector. This study 
mentioned that a perfectly rational individual joins the military sector if the pay
differential between the sectors exceeds the preference for civilian life. \textcite{warner_chapter_1995} 
also note that the USA and its allies spend a significant amount of 
their defence budgets on military personnel. This means that GDP and defence spending, 
along with their dynamics, could reflect employment opportunities in the civilian and 
military sectors and in turn, be correlated with choices to enlist and stay in the military.
Although \textcite{backstrom_are_2019} researched mostly unemployment rates, he highlights 
the fact that a stronger civilian economy increases the difficulty of recruiting new military 
labour force. On the other hand, \textcite{holcner_military_2021} used a multiple linear 
regression model to find that the annual 
increase in GDP, indicating a growing economy, had a positive impact on the recruitment to the 
Czech Armed Forces. The annual increase in defence expenditure was also found to be 
correlated with a higher number of military recruits in that study.

The U.S Department of Defense has a benchmark of at least 90\% of new military recruits 
having secondary education or higher \parencite{cnas_resources_and_force_readiness_division_fiscal_nodate}.
This emphasises the fact that the military is looking for educated recruits. \textcite{asoni_rich_2013} 
also argue that the transition to a smaller and technologically advanced military has 
made the recruitment process more selective and less likely for an individual to 
be allowed in the military without a high school degree. \textcite{elster_study_1982} 
found that the four-year retention rates were higher amongst U.S military recruits 
with a high-school degree, compared to those with lower educational attainment. Additionally,
this study used grouped data in regression analysis to study the attrition rates from the military 
and found high school graduates to have lower attrition rates than those with lower 
educational attainment.
However, this study was conducted with data from the 1970s, which may not reflect the current 
socioeconomic or military environment. 
In contrast, a more recent study by \textcite{hof_quality_2023} found that in the 
Dutch Armed Forces recruits training to become officers, who have a higher level of 
secondary education prior to enlistment, show higher intentions to quit basic training, 
than those training to become noncommissioned officers, who have a lower level of 
secondary education. This study hypothesised that recruits with a better educational 
background could believe that they have better opportunities in the civilian labour market.
\textcite{asch_cash_2010} included a variable for the percentage of high school graduates,
who are enrolled in college, in their regression model, but found no statistically significant
relationship with the number of high-quality enlistment contracts, further reinforcing the 
mixed results on the impact of educational attainment on military personnel.

The reviewed literature predominantly relies on quantitative methods, commonly using 
panel regression models to estimate the impact of socioeconomic variables on 
military recruitment and retention. Fixed-effects models are often utilised to account for 
heterogeneity across geographical units, while random-effects and time-controlled fixed-effects
models are also employed in some studies. These methods are appropriate for this 
research, as they can control for temporal and regional variation, allowing 
researchers to isolate the effects of different factors.

Overall, the literature suggests complex and sometimes even contradictory relationships 
between the socioeconomic factors and military recruitment and retention. While 
Many studies find unemployment rates to be positively correlated 
with enlistment, implying that individuals are more likely to join the military, when 
civilian employment opportunities are scarce, while others have also reported the opposite. 
These inconsistencies may reflect differences in national 
contexts or time periods. Similarly, while economic theory might suggest that 
stronger civilian economies may reduce the appeal of military employment and that 
defence spending should increase it, the empirical evidence is not consistent.
This could indicate the need to rethink these assumptions or consider additional 
factors. Education also exhibits mixed results, with higher educational attainment 
improving eligibility, while also potentially increasing attrition, as better educated recruits 
may have better civilian opportunities. Socioeconomic conditions and 
military structures differ between nations, which may lead to these inconsistencies.
The mixed findings, along with most studies 
focusing on single countries or isolated variables or outdated data, underscore the 
need for comparative research across NATO countries, to provide generalisable and 
consistent insights into military labour supply dynamics. 

